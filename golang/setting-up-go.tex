\documentclass[8pt]{article}
\usepackage{hyperref}
\title{Getting Going}
\date{}
\begin{document}
\maketitle
\section{Installing Go}

Go is a friendly language, but before we use it we (unsurprisingly) have to install it.
Go is open source, which means you can read the source code as much and as often as you desire.

A guide to installation is provided here: \url{https://golang.org/doc/install}.

We recommend you use vscode, as this is both simplest to install and has the richest set of
features. It has multi-language support, but for setting it up with Go in particular, the
following links are helpful:
\begin{itemize}
  \item \url{https://code.visualstudio.com}
  \item \url{https://github.com/Microsoft/vscode-go}
  \item  \url{https://marketplace.visualstudio.com/items?itemName=ms-vscode.Go}
\end{itemize}

(If you have strong editor preferences, you'll need to set your path and gopath in your 
\texttt{.bashrc} or similar. Check if the \texttt{GOPATH} has been set
correctly by running the command \texttt{go env} in a terminal, and inspecting the \texttt{GOPATH} variable.)

\section{Using Go}

Go uses packages like Java, but to import you just need the package name, not the whole path to the directory.
This is made possible because all the packages live in your \texttt{GOPATH} (which will be set by vscode).

An introduction to Go for Java Programmers is found at \url{https://talks.golang.org/2015/go-for-java-programmers.slide#1},
and there is another at
\url{https://dzone.com/articles/quick-go-lang-for-java-developers}.

%
%The main function is the entrypoint to a package you've named `main`, and it is executed using `go run`.
%The main argument must take no argument and return no values. The function declaration is then the
%minimum possible:
%
%```
%func main() {
%  // function goes here
%}
%```
%
%Different functions can take arguments and have (even multiple) return values.
%The arguments go in the parentheses, and the type of the return variable comes after:
%
%
%```
%func funWithArgs(arg1 int, arg2 int) int {
%  // function goes here
%  return x
%}
%```
%
%The return variables generally aren't named in the function declaration.
%Returning more than one variable is done as follows:
%
%
%```
%func funWithReturns(arg1 int) (int, int) {
%  // function goes here
%  return x, y
%}
%```
%
%Methods are called on a type, and this appears in parentheses before the function name:
%
%```
%func (T type) someMethod(arg int) (int, int, int, int) {
%  // function goes here
%  return T.x, T.y, T.u, T.v
%}
%```
%
\end{document}
